\documentclass[]{article}
\usepackage{lmodern}
\usepackage{amssymb,amsmath}
\usepackage{ifxetex,ifluatex}
\usepackage{fixltx2e} % provides \textsubscript
\ifnum 0\ifxetex 1\fi\ifluatex 1\fi=0 % if pdftex
  \usepackage[T1]{fontenc}
  \usepackage[utf8]{inputenc}
\else % if luatex or xelatex
  \ifxetex
    \usepackage{mathspec}
  \else
    \usepackage{fontspec}
  \fi
  \defaultfontfeatures{Ligatures=TeX,Scale=MatchLowercase}
\fi
% use upquote if available, for straight quotes in verbatim environments
\IfFileExists{upquote.sty}{\usepackage{upquote}}{}
% use microtype if available
\IfFileExists{microtype.sty}{%
\usepackage{microtype}
\UseMicrotypeSet[protrusion]{basicmath} % disable protrusion for tt fonts
}{}
\usepackage[margin=1in]{geometry}
\usepackage{hyperref}
\hypersetup{unicode=true,
            pdftitle={Semantic Overlap Computation},
            pdfauthor={Kanishka Misra, Hemanth Devarapalli},
            pdfborder={0 0 0},
            breaklinks=true}
\urlstyle{same}  % don't use monospace font for urls
\usepackage{graphicx,grffile}
\makeatletter
\def\maxwidth{\ifdim\Gin@nat@width>\linewidth\linewidth\else\Gin@nat@width\fi}
\def\maxheight{\ifdim\Gin@nat@height>\textheight\textheight\else\Gin@nat@height\fi}
\makeatother
% Scale images if necessary, so that they will not overflow the page
% margins by default, and it is still possible to overwrite the defaults
% using explicit options in \includegraphics[width, height, ...]{}
\setkeys{Gin}{width=\maxwidth,height=\maxheight,keepaspectratio}
\IfFileExists{parskip.sty}{%
\usepackage{parskip}
}{% else
\setlength{\parindent}{0pt}
\setlength{\parskip}{6pt plus 2pt minus 1pt}
}
\setlength{\emergencystretch}{3em}  % prevent overfull lines
\providecommand{\tightlist}{%
  \setlength{\itemsep}{0pt}\setlength{\parskip}{0pt}}
\setcounter{secnumdepth}{0}
% Redefines (sub)paragraphs to behave more like sections
\ifx\paragraph\undefined\else
\let\oldparagraph\paragraph
\renewcommand{\paragraph}[1]{\oldparagraph{#1}\mbox{}}
\fi
\ifx\subparagraph\undefined\else
\let\oldsubparagraph\subparagraph
\renewcommand{\subparagraph}[1]{\oldsubparagraph{#1}\mbox{}}
\fi

%%% Use protect on footnotes to avoid problems with footnotes in titles
\let\rmarkdownfootnote\footnote%
\def\footnote{\protect\rmarkdownfootnote}

%%% Change title format to be more compact
\usepackage{titling}

% Create subtitle command for use in maketitle
\newcommand{\subtitle}[1]{
  \posttitle{
    \begin{center}\large#1\end{center}
    }
}

\setlength{\droptitle}{-2em}

  \title{Semantic Overlap Computation}
    \pretitle{\vspace{\droptitle}\centering\huge}
  \posttitle{\par}
    \author{Kanishka Misra, Hemanth Devarapalli}
    \preauthor{\centering\large\emph}
  \postauthor{\par}
      \predate{\centering\large\emph}
  \postdate{\par}
    \date{1/22/2019}

\usepackage[utf8]{inputenc}
\usepackage{fontspec}

\setmainfont[
  Ligatures=TeX,
  Extension=.otf,
  BoldFont=cmunbx,
  ItalicFont=cmunti,
  BoldItalicFont=cmunbi,
]{cmunrm}

\usepackage{polyglossia}
\setdefaultlanguage{english}
\setotherlanguage{russian}

\newcommand{\RU}[1]{\foreignlanguage{russian}{#1}}
\usepackage{booktabs}
\usepackage{longtable}
\usepackage{array}
\usepackage{multirow}
\usepackage[table]{xcolor}
\usepackage{wrapfig}
\usepackage{float}
\usepackage{colortbl}
\usepackage{pdflscape}
\usepackage{tabu}
\usepackage{threeparttable}
\usepackage{threeparttablex}
\usepackage[normalem]{ulem}
\usepackage{makecell}

\begin{document}
\maketitle

\hypertarget{learner-errors-in-english}{%
\section{Learner Errors in English}\label{learner-errors-in-english}}

In this example, we consider an error annotated corpus of 1244 short
essays, each written by a unique learner of english belonging to one of
16 different first languages (L1).

Since we are covering semantic errors, we extract the errors made in
content words (adjectives (J), nouns (N), verbs (V) and adverbs (Y)).

Let's look at an example by a learner whose L1 is Russian:

While this example contains many other errors, we only focus on one for
this analysis. The error word is in \textbf{bold}.

\hypertarget{original-answer}{%
\subsection{Original Answer:}\label{original-answer}}

\emph{Dear Jane Clark, I am writing to you to give some opinions of mine
abut The International Arts Festival. It was a great idea and I had the
pleasure of resting and relaxing, plus of \textbf{getting} some
important knowledge from it. On the other hand there were some
disadvantages: 1. Stars and artists wee only from six countries 2. Some
concert halls were too small. So it was impossible to get tickets there.
3. I think that there were not enough plays and films. To sum up I want
to tell you bout my suggestions for next year's festival. It would be
wonderful to buy some books or programms with signatures of all the
stars and artists taking part in the festival. Also I think that dance
shows should include different styles of dancing (e.g.~national dancing
of all the countries) to impress audience and not to bore with the
similar scenes . I hope this letter will help you in organying Yours
faithfully}

\hypertarget{correct-answer}{%
\subsection{Correct Answer:}\label{correct-answer}}

\emph{Dear Jane Clark, I am writing to you to give you some opinions of
mine about The International Arts Festival. It was a great idea and I
had the pleasure of resting and relaxing, plus of \textbf{acquiring}
some important knowledge from it. On the other hand there were some
disadvantages: 1. The stars and artists were from only six countries .
2. Some concert halls were too small. So it was impossible to get
tickets there. 3. I don't think that there were enough plays and films.
To sum up I want to tell you about my suggestions for next year's
festival. It would be wonderful to buy some books or programmes with the
signatures of all the stars and artists taking part in the festival.
Also I think that the dance shows should include different styles of
dancing (e.g.~the national dance of all the countries) to impress the
audience and not to bore with the similar scenes . I hope this letter
will help you with organizing the festival . Yours sincerely}

\hypertarget{analysis}{%
\subsection{Analysis}\label{analysis}}

From the above short essay responses, we get the incorrec and
replacement annotations

Incorrect word in English: \textbf{getting} Replacement as annotated:
\textbf{acquiring}

We then take this (incorrect, correct) word pair and translate it into
the person's L1 language using the Microsoft Azure Text translation API
\footnote{\url{https://docs.microsoft.com/en-us/azure/cognitive-services/Translator/reference/v3-0-reference}}.
We use this API because unlike Google's Cloud API for translations, it
provides us with alternate pronunciations which prove beneficial while
querying for words that can be expressed with different parts of speech.

Translated form: \textbf{получение} Translated replacement:
\textbf{приобретения}

\hypertarget{semantic-overlap}{%
\subsection{Semantic Overlap}\label{semantic-overlap}}

We then use the fasttext english and russian word vectors (300
dimensions) to calculate the 10 nearest neighbors for the
\emph{(incorrect, correct)} pair in their respective language vector
spaces.

The 10 nearest neighbors for each of these words are shown in table 1.

\begin{table}[!h]

\caption{\label{tab:unnamed-chunk-4}10 nearest neighbors of the words in the given example.}
\centering
\begin{tabular}[t]{llll}
\toprule
\multicolumn{2}{c}{English} & \multicolumn{2}{c}{Russian} \\
\cmidrule(l{2pt}r{2pt}){1-2} \cmidrule(l{2pt}r{2pt}){3-4}
\textbf{getting} & \textbf{acquiring} & \textbf{получение} & \textbf{приобретения}\\
\midrule
gettting & Acquiring & Получение & покупки\\
gettng & reacquiring & предоставление & приобритения\\
geting & obtaining & -получение & продажи\\
get & acquire & приобретение & получения\\
gettign & re-acquiring & неполучение & приобретении\\
\addlinespace
gettiing & acquired & получения & приобретение\\
gettig & procuring & лучение & перепродажи\\
got & acquring & наполучение & преобретения\\
getitng & acquisition & получении & приобретению\\
gotten & owning & пοлучение & использования\\
\bottomrule
\end{tabular}
\end{table}

To compute the semantic overlap between \textbf{i} and \textbf{c} in a
given language, we take each word and compute its partial overlap to the
other word. The partial overlap of a word \emph{x} with word \emph{y} is
computed as:

\[
  PO(x, y) = \frac{1}{k} \sum_{y' \in NN_k(y)} cos(x, y')
\]

Where \(k\) is the number of nearest neighbors, here 10 and \(NN_k(y)\)
is the nearest neighbor function to print the k nearest neighbors of the
word \(y\) based on cosine similarity. Thus, we get the partial overlap
of \textbf{i} with \textbf{c} and \textbf{c} with \textbf{i}. We then
compute the Semantic Error Overlap, \(SEO\) by taking the mean of the
two partial overlaps \(PO(i, c)\) and \(PO(c, i)\).

\begin{figure}[h]
  \caption {Semantic Overlap between incorrect and correct words for a language}
  \centering
    \includegraphics[width=1\textwidth]{../figures/semantic_neighborhood.png}
\end{figure}

Based on the example:

For \textbf{getting} and \textbf{acquiring},

We compute \(PO(getting, acquiring)\) and \(PO(acquiring, getting)\) as:

\texttt{getting\ with\ {[}Acquiring,\ reacquiring,\ obtaining,\ acquire,\ re-acquiring,\ acquired,\ procuring,\ acquring,\ acquisition,\ owning{]}\ =\ 0.32476866}

and

\texttt{acquiring\ with\ {[}gettting,\ gettng,\ geting,\ get,\ gettign,\ gettiing,\ gettig,\ got,\ getitng,\ gotten{]}\ =\ 0.27844474}

Taking the average, we have: \(SEO(getting, acquiring)\)
\textasciitilde{} 0.302

Similarly, we compute \(PO\)\emph{(получение, приобретения)} =
0.41347638 and \(PO\)\emph{(приобретения, получение)} = 0.41594142 to
get \(SEO\)\emph{(получение, приобретения)} \textasciitilde{} 0.415


\end{document}
